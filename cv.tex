\documentclass[10pt,a4paper,final]{article}
\usepackage[utf8]{inputenc}
\usepackage[left=2cm,right=2cm,top=2cm,bottom=2cm]{geometry}
\usepackage[hidelinks]{hyperref}
\urlstyle{same}
\usepackage{amsmath}
\usepackage{amsfonts}
\usepackage{amssymb}
\usepackage{graphicx}
\usepackage{array}
\usepackage[table]{xcolor}
\usepackage{tabularx}
\usepackage{multirow}
\usepackage{colortbl}
% Definitions of colours used in seaborn for use in latex
\definecolor{seaborn_bg_grey}             {HTML}{eaeaf2}
\definecolor{seaborn_bg_grey_half}        {HTML}{f4f4f8}
 
\definecolor{seaborn_blue}                {HTML}{4c72b0}
\definecolor{seaborn_green}               {HTML}{55a868}
\definecolor{seaborn_red}                 {HTML}{c44e52}
\definecolor{seaborn_magenta}             {HTML}{8172b2}
\definecolor{seaborn_yellow}              {HTML}{ccb974}
\definecolor{seaborn_cyan}                {HTML}{64b5cd}
 
\definecolor{seaborn_muted_blue}          {HTML}{4878cf}
\definecolor{seaborn_muted_green}         {HTML}{6acc65}
\definecolor{seaborn_muted_red}           {HTML}{d65f5f}
\definecolor{seaborn_muted_magenta}       {HTML}{b47cc7}
\definecolor{seaborn_muted_yellow}        {HTML}{c4ad66}
\definecolor{seaborn_muted_cyan}          {HTML}{77bedb}
 
\definecolor{seaborn_pastel_blue}         {HTML}{92c6ff}
\definecolor{seaborn_pastel_green}        {HTML}{97f0aa}
\definecolor{seaborn_pastel_red}          {HTML}{ff9f9a}
\definecolor{seaborn_pastel_magenta}      {HTML}{d0bbff}
\definecolor{seaborn_pastel_yellow}       {HTML}{fffea3}
\definecolor{seaborn_pastel_cyan}         {HTML}{b0e0e6}
 
\definecolor{seaborn_bright_blue}         {HTML}{003fff}
\definecolor{seaborn_bright_green}        {HTML}{03ed3a}
\definecolor{seaborn_bright_red}          {HTML}{e8000b}
\definecolor{seaborn_bright_magenta}      {HTML}{8a2be2}
\definecolor{seaborn_bright_yellow}       {HTML}{ffc400}
\definecolor{seaborn_bright_cyan}         {HTML}{00d7ff}
 
\definecolor{seaborn_dark_blue}           {HTML}{001c7f}
\definecolor{seaborn_dark_green}          {HTML}{017517}
\definecolor{seaborn_dark_red}            {HTML}{8c0900}
\definecolor{seaborn_dark_magenta}        {HTML}{7600a1}
\definecolor{seaborn_dark_yellow}         {HTML}{b8860b}
\definecolor{seaborn_dark_cyan}           {HTML}{006374}
 
\definecolor{seaborn_colorblind_blue}     {HTML}{0072b2}
\definecolor{seaborn_colorblind_green}    {HTML}{009e73}
\definecolor{seaborn_colorblind_red}      {HTML}{d55e00}
\definecolor{seaborn_colorblind_magenta}  {HTML}{cc79a7}
\definecolor{seaborn_colorblind_yellow}   {HTML}{f0e442}
\definecolor{seaborn_colorblind_cyan}     {HTML}{56b4e9}
 

\usepackage[{sf,bf}]{titlesec}
\usepackage{layouts}
\usepackage{calc}
\def\arraystretch{1.5} % Add nicer padding within tables
\usepackage{placeins} % FloatBarrier
\renewcommand{\familydefault}{\sfdefault} % Serif
% Setting up tables to by default alternate grey and white
\rowcolors{2}{seaborn_bg_grey}{white}
\usepackage{longtable}

\author{Edward Linscott}
\title{Marie Curie Part B2}

\begin{document}
\begin{table}[t!]
\begin{tabularx}{\textwidth}{l X l X}
\multicolumn{4}{l}{\cellcolor{seaborn_red}} \\[-1.5em]
\multicolumn{4}{l}{\cellcolor{seaborn_red}%
\Huge\textbf{\textcolor{white}{Curriculum Vitae}%
}} \\
\textit{Name         }  & Edward Baxter Linscott & & \\
\textit{Nationality  }  & New Zealander          & \textit{Email  } & ebl27@cam.ac.uk\\
\textit{Date of birth}  & 26/11/1991             & \textit{Website} & \url{https://elinscott.github.io/}\\

\end{tabularx}
\end{table}

\begin{table}[t!]
\begin{tabularx}{\textwidth}{l l l l l}
\multicolumn{5}{l}{\cellcolor{seaborn_blue}\large\textbf{\textcolor{white}{Education}}} \\
\rowcolor{white}\textbf{start date} & \textbf{end date} & \textbf{degree/course} & \textbf{institution} & \textbf{grade}\\
%
% PhD
%
\rowcolor{seaborn_bg_grey}
1 Oct 2015 & present  & PhD in Physics (in progress) & University of Cambridge & \\
\multicolumn{5}{X}{\rowcolor{seaborn_bg_grey}
   \textit{Title:} ``Describing Correlation Effects in Biological Systems"
}\\
\multicolumn{5}{X}{\rowcolor{seaborn_bg_grey}
   \textit{Supervisors:} Prof. Mike Payne and Dr. Daniel Cole
}\\
\multicolumn{5}{X}{\rowcolor{seaborn_bg_grey}
   \textit{Funding:} Cambridge-Rutherford Memorial Scholarship
}\\
\multicolumn{5}{X}{\rowcolor{seaborn_bg_grey}
Many key reactions in biology are performed by metalloproteins. These systems are a challenge to accurately simulate due to two contrasting reasons. Firstly, the strong correlation present due to the transition metal atoms requires more accurate theories than semi-local density-functional theory. And secondly the ambient protein environment requires explicit treatment of thousands of atoms at the quantum-mechanical level. Over the course of my PhD I developed new linear-response theory for determining Hubbard and Hund's parameters from first principles for DFT\,+\,\emph{U} calculations. I also developed a dynamical mean-field theory module for the linear scaling package ONETEP. I applied these tools to study (a) photodissociation of carboxy-heme (b) the electronic structure of hemocyanin, and (c) water-to-oxygen conversion performed by the oxygen evolving complex.
} \\
%
% Masters
%
\rowcolor{white}
1 Oct 2014
& 30 Sep 2015
& \multicolumn{1}{m{0.26\textwidth}}{MPhil in Scientific \mbox{Computing}}
& University of Cambridge
& Distinction \\
\rowcolor{white}
\multicolumn{5}{X}{
   \textit{Title:} ``Strong Correlation Effects in the Electronic Structure of the Photosystem II Complex"
}\\
\rowcolor{white}
\multicolumn{5}{X}{
   \textit{Supervisors:} Prof. Mike Payne and Dr. Daniel Cole
}\\
\rowcolor{white}
\multicolumn{5}{X}{
   \textit{Funding:} EPSRC
}\\
\rowcolor{white}
\multicolumn{5}{X}{
Designed to lead into my PhD, my masters project motivated the need for models of the oxygen-evolving complex (OEC) that are thousands of atoms in size, and demonstrated that such calculations are feasible with the linear scaling density functional theory code ONETEP. The thesis also explored the DFT\,+\,\emph{U} as a method for treating the correlation present in the OEC core.
} \\
%
% Hons
%
\rowcolor{seaborn_bg_grey} 25 Feb 2013 & 4 Nov 2013 & BSc (Hons) in Physics & University of Otago & First Class\\ 
\multicolumn{5}{X}{\rowcolor{seaborn_bg_grey}
    \textit{Title:} ``Non-zero Temperature Theory for Ultra-Cold Dipolar Bose Gases"
}\\
\multicolumn{5}{X}{\rowcolor{seaborn_bg_grey}
   \textit{Supervisor:} Prof. P. Blair Blakie
}\\
\multicolumn{5}{X}{\rowcolor{seaborn_bg_grey}
This one-year honours programme comprised of taught courses in physics and a research project. My research project explored the effects of temperature on the behaviour of quasi-2D dipolar Bose-Einstein condensates; that is, BECs whose atoms (a) interact with an appreciable magnetic dipole-dipole moment and (b) are contained within a 2D optical trap. (See the ``Professional Experience" section for the outcome of this research.)
}\\
%
% BSc
%
\rowcolor{white}
1 Mar 2010 & 10 Nov 2012 &
\multicolumn{1}{m{0.26\textwidth}}{BSc in Mathematics and Physics} &
\multicolumn{1}{m{0.25\textwidth}}{University of \mbox{Cambridge} and University of \mbox{California} Berkeley} & Straight A\textsuperscript{+}'s\\
\rowcolor{white}
\multicolumn{5}{X}{
A three-year Bachelor's degree with a double-major in mathematics and physics. I spent the final semester of this degree on exchange at Berkeley.
} \\
%
% STAR
%
\rowcolor{seaborn_bg_grey} 1 Feb 2009 & 30 Nov 2009 & STAR Mathematics course & University of Canterbury & A\textsuperscript{+} \\
\multicolumn{5}{X}{\rowcolor{seaborn_bg_grey} 
A course equivalent to Canterbury's first-year mathematics programme, but run specifically for accelerated high-school students
}
\end{tabularx}
\end{table}
%}


\begin{table}[t!]
\begin{tabularx}{\textwidth}{l l l l}
\multicolumn{4}{l}{\cellcolor{seaborn_blue}\large\textbf{\textcolor{white}{Professional Experience}}} \\
\rowcolor{white}
\textbf{start date} & \textbf{end date} & \textbf{job title} & \textbf{institution} \\
\rowcolor{seaborn_bg_grey}
1 Mar 2013 & 31 July 2013 & Research Assistant & University of Otago \\
\rowcolor{seaborn_bg_grey}
\multicolumn{4}{X}{
I was briefly employed as a Research Assistant at the University of Otago, where I continued the work from my honours year studying the behaviour of dipolar Bose gases. This work resulted in a publication where we predicted an instability of dipolar BECs in regions of experimental interest.
} \\
\rowcolor{white}
21 Nov 2011 & 30 Jan 2012 & Summer Studentship & University of Otago \\
\rowcolor{white}
\multicolumn{4}{X}{
I spent a summer in a studentship with an engineering/physics/medical research group developing new computed tomography (CT) scanning technology. I was involved in developing software that allowed the research team to quantify the quality of their images (and hence assess the performance of their machines during development).} \\
\end{tabularx}
\end{table}

\begin{table}[t!]
\begin{tabularx}{\textwidth}{X l}
\rowcolor{seaborn_blue}
\multicolumn{2}{l}{\large\textcolor{white}{\textbf{Publications}}} \\
EBL and P.\,B.\,Blakie, \textit{Thermally activated local collapse of a flattened dipolar condensate}. Phys. Rev.~A 90 (2014) 053605 & 
\multicolumn{1}{m{0.2\textwidth}}{4 citations; selected as an Editors' Suggestion} \\
EBL, D.\,J.\,Cole, M.\,C.\,Payne, and D.\,D.\,O'Regan, \textit{The role of spin in the calculation of Hubbard U and Hund’s J parameters from first principles}. & under review \\
EBL, D.\,J.\,Cole, and C.\,Weber, \textit{Introducing TOSCAM: uniting dynamical mean field theory and linear-scaling density functional theory, and its application to carboxymyoglobin}. & in preparation \\
M.\,A.\,al-Badri, EBL, D.\,J.\,Cole, and C.\,Weber, \textit{Renormalization of hemocyanin-ligand binding energetics by quantum many-body effects}. & in preparation \\
\end{tabularx}
\end{table}

\begin{table}[t!]
\begin{tabularx}{\textwidth}{l m{0.9\textwidth}}
\rowcolor{seaborn_blue}
\multicolumn{2}{l}{\large\textcolor{white}{\textbf{Scholarships, Awards, And Prizes}}} \\
\multicolumn{2}{l}{\textbf{\textit{Postgraduate}}} \\
{2016} & \textit{Poster prize} at CCP9 Young Researchers' Event \\
{2014} & \textit{LB Wood Scholarship} to supplement an existing scholarship for postgraduate study in Britain \\
{2013} & \textit{Cambridge--Rutherford Memorial Scholarship} for doctorate studies at the University of Cambridge \\
& \textit{Douglass D. Crombie Award in Physics} for an Otago graduate embarking on doctoral studies overseas \\
\multicolumn{2}{l}{\textbf{\textit{Undergraduate}}} \\
{2013} & \textit{Prince of Wales Award} for the most outstanding student completing an undergraduate degree at the University of Otago in 2013\\
{2012} & \textit{University of Otago Prestige Scholarship in Science} to support honours study\\
       & \textit{Beverley Bursary} for study towards physics honours (2011-2013)\\
       & \textit{The Alumni of the University of Otago in America Inc. Award} to support an academic exchange to the United States\\
{2011} & \textit{Robert Jack / Institute of Physics Prize} for the top student in 200-level physics\\
       & \textit{Gloria Olive Memorial Prize in Mathematics} for the top student in 300-level mathematics\\
\end{tabularx}
\end{table}
\begin{table}[t!]
\begin{tabularx}{\textwidth}{l m{0.9\textwidth}}
{2010} & \textit{Alumni Annual Appeal Scholarship} for study at the University of Otago\\
       & \textit{University of Otago Dux Scholarship} for study at the University of Otago\\
       & \textit{University of Canterbury Dux Scholarship} for study at the University of Canterbury\\
       & \textit{University of Canterbury Science Scholarship} for study at the University of Canterbury\\
       & \textit{Department of Mathematics and Statistics Scholarship} for study at the University of Canterbury\\
       & \textit{Robert Jack / Institute of Physics Prize} for the top student in 100-level physics\\
       & \textit{R. J. T. Bell Prize} for the top student in 200-level mathematics\\
       & \textit{New Zealand Institute of Chemistry Prize} for the top student in CHEM111: Molecular Architecture \\
\multicolumn{2}{l}{\textbf{\textit{High School}}} \\
{2009} & \textit{Dux} of Cashmere High School\\
& \textit{NCEA Scholarships} in Calculus, Physics, and Chemistry\\
& Top Year 13 student in English, Physics, and Chemistry\\
{2008} & \textit{McCombs Scholarship} for all round excellence and contributions to Cashmere High\\
& Top Year 12 student in English, Art, Physics, Chemistry, and Year 13 Calculus\\
& \textit{NCEA Scholarship} in Calculus
\end{tabularx}
\end{table}

\begin{table}[!t]
\begin{tabularx}{\textwidth}{l X l l}
\rowcolor{seaborn_blue}
\multicolumn{4}{l}{\large\textcolor{white}{\textbf{Conferences, Schools, and Workshops}}} \\
\textbf{Year}& \textbf{Event} & \textbf{Location} & \textbf{Contribution} \\
{2018}
& Autumn School on Correlated Electrons & J\"ulich & Poster \\
& CCP9 Young Researchers Event & York & Poster \\
& DPG March Meeting & Berlin & \\
& CDT student-run seminar series & Cambridge & Invited talk \\
{2017}
& Autumn School on Correlated Electrons & J\"ulich & Poster \\
& CCP-BioSim Conference: Frontiers of Biomolecular Simulation & Southampton & Poster \\
& New Generation in Strongly Correlated Electron Systems & Barcelona & Contributed talk \\
& ONETEP Masterclass & Warwick & Tutor \\
& Workshop on Localisation in Quantised Systems & London & \\ 
& CCP9 Young Researchers Event & Cambridge & \\ 
{2016} & Physics by the Lake & Windsor & Poster \\
& CCP9 Young Researchers Event & York & Poster \\
& ``Programming: Modern Fortran" UCS workshop & Cambridge & \\
{2015} & Psi-K conference & San Sebastian & Poster \\
& ONETEP Masterclass & Cambridge &
\end{tabularx}
\end{table}
%
\begin{table}[!t]
\begin{tabularx}{\textwidth}{l l l}
\rowcolor{seaborn_blue}
\multicolumn{3}{l}{\large\textcolor{white}{\textbf{Skills}}} \\
\rowcolor{white}
\textbf{\textit{Programming}} & \\
\rowcolor{white}
Used daily & Fortran, Python, Bash & \\ 
\rowcolor{white}
Used monthly & MPI, OpenMP & \\
\rowcolor{white}
Some experience & C++, MATLAB, CUDA & \\
\rowcolor{white}
\multicolumn{3}{X}{I am a developer of \href{https://www.onetep.org}{ONETEP}, a commercially available scientific DFT code, and TOSCAM, a publically available DMFT code} \\
\rowcolor{white}
\multicolumn{3}{X}{I participated in \href{https://hashcode.withgoogle.com/}{Google Hash Code 2018}. I sporadically compete on \href{https://projecteuler.net/}{Project Euler} and \href{https://www.codingame.com/home}{CodinGame}} \\
\rowcolor{seaborn_bg_grey} 
\multicolumn{3}{X}{\textbf{\textit{Packages and Software}}} \\
\rowcolor{seaborn_bg_grey}
Used daily & ONETEP, TOSCAM, vim, \LaTeX, SLURM, git & \\
\rowcolor{seaborn_bg_grey}
Used monthy & PyMol, VMD, ASE & \\
\rowcolor{seaborn_bg_grey}
Some experience & CASTEP, Siesta, Maestro & \\
\rowcolor{white}
\textbf{\textit{Teaching}} \\
\rowcolor{white}
2017 
& \multicolumn{1}{m{0.5\textwidth}}{Supervised ten third-year students for thermal and statistical physics}
& University of Cambridge \\
\rowcolor{white}
2016
& \multicolumn{1}{m{0.5\textwidth}}{Supervised nine second-year students for experimental methods, oscillations, waves, optics, quantum mechanics, and condensed matter}
& University of Cambirdge \\
\rowcolor{white}
2015
& \multicolumn{1}{m{0.5\textwidth}}{Supervised nine first-year students for physics}
& University of Cambridge \\
\rowcolor{white}
2013
& \multicolumn{1}{m{0.5\textwidth}}{Lab demonstrator, university tutor, and private tutor for first-year biological physics}
& University of Otago \\
\rowcolor{seaborn_bg_grey}
\textbf{\textit{Mentoring}} & & \\
\rowcolor{seaborn_bg_grey}
\multicolumn{3}{X}{During the course of my PhD I have provided support to two students.} \\
\rowcolor{seaborn_bg_grey}
\multicolumn{3}{X}{I have spent a significant amount of time with M.\,A.\,Al-Badri (Masters, and then PhD student from King's College London), teaching him about DMFT and working with him on DMFT calculations on hemocyanin. I have hosted him in Cambridge twice, and visited him at KCL periodically.} \\
\rowcolor{seaborn_bg_grey}
\multicolumn{3}{X}{I have been the local port-of-call for S. Mansour (PhD student, Cambridge) for technical ONETEP support.}
\end{tabularx}
\end{table}
%

\begin{table}[!t]
\begin{tabularx}{\textwidth}{X}
\rowcolor{seaborn_blue}
\multicolumn{1}{l}{\large\textcolor{white}{\textbf{Interests}}} \\
I am a violinist, and have been a member (and concertmaster) of many orchestras including the New Zealand Symphony Secondary Schools Orchestra, the Christchurch Youth Orchestra, and the Cambridge Musical Society Symphony Orchestra. I love the outdoors and am an avid runner, cyclist, camper, and hiker.%  I enjoy pub quizzes (having run several for my college's postgraduate community)
\end{tabularx}
\end{table}
%
\begin{table}[!t]
\begin{tabularx}{\textwidth}{X X}
\rowcolor{seaborn_blue}
\multicolumn{2}{l}{\large\textcolor{white}{\textbf{Referees}}} \\
Prof. Mike Payne                   & Prof. P. Blair Blakie \\
Room 528, Mott Building            & Room 420, Science III\\ 
Cavendish Laboratory               & Department of Physics \\
University of Cambridge            & University of Otago \\
19 J J Thomson Avenue              & 362 Leith St \\
Cambridge CB3 0HE                  & Dunedin 9016 \\
United Kingdom                     & New Zealand \\
mcp1@cam.ac.uk                     & blair.blakie@otago.ac.nz \\
+44 (0)1223 337254                 & +64 (0)3 479 4114\\
\end{tabularx}
\end{table}
\FloatBarrier

\end{document}
